\documentclass[utf8]{ctexart} %中文文档类
\usepackage[left=2.50cm, right=2.50cm, top=2.50cm, bottom=2.50cm]{geometry} % 页边距
\usepackage{indentfirst} % 首行缩进
\usepackage{fancyhdr} % 设置页眉、页脚
\renewcommand\headrulewidth{0pt}% 页眉与正文之间的水平线粗细
\usepackage{ctexcap} % 标题是中文的
\usepackage{helvet} % 用来指定beamer使用的字体
\usepackage{hyperref} % bookmarks 
\usepackage{multicol} % 分栏
\usepackage{float}
\usepackage{amsmath, amsfonts, amssymb} % 数学公式、符号
\usepackage[english]{babel} % 数学公式标准
\usepackage{bm} % 加粗方程字体 
\usepackage{float}
\usepackage{listings}
\usepackage{xcolor}
\usepackage{graphicx} % 图片 
\usepackage{url} % 超链接 

\usepackage{multirow} % 表格
\usepackage{booktabs} % 三线表
\usepackage{longtable} % 长表格

%\usepackage{algorithm} % 算法或伪代码
%\usepackage{algorithmic} % 算法或伪代码
\usepackage[linesnumbered,lined,boxed,commentsnumbered]{algorithm2e}

\usepackage{enumitem} % 枚举环境宏包

%\renewcommand{\algorithmicrequire}{ \textbf{Input:}} 
%\renewcommand{\algorithmicensure}{ \textbf{Initialize:}} 
%\renewcommand{\algorithmicreturn}{ \textbf{Output:}} %算法格式

\newtheorem{theorem}{\indent 定理}[section]
\newtheorem{lemma}[theorem]{\indent 引理}
\newtheorem{proposition}[theorem]{\indent 命题}
\newtheorem{corollary}[theorem]{\indent 推论}
\newtheorem{definition}{\indent 定义}[section]
\newtheorem{example}{\indent 例}[section]
\newtheorem{remark}{\indent 注}[section]
\newenvironment{solution}{\begin{proof}[\indent\bf 解]}{\end{proof}}
\renewcommand{\proofname}{\indent\bf 证明}

\pagestyle{fancy} \lhead{} \chead{} \lfoot{} \cfoot{} \rfoot{}
\pagestyle{plain}
\hypersetup{colorlinks, bookmarks, unicode} % unicode 
 
\title{\textbf{Learn Computer System}}
\author{\bf Chirality}
\date{}
\hypersetup{
	colorlinks=true,
	linkcolor=black
} % 使目录为黑色字
\lstset{
	numbers=left, % 显示行号
	numberstyle=\tiny, % 行号字体
	keywordstyle=\color{blue!70}, % 关键字颜色
	commentstyle=\color{red!50!green!50!blue!50}, % 注释颜色
	frame=shadowbox, % 为代码块添加阴影框
	rulesepcolor=\color{red!20!green!20!blue!20}, % 阴影框颜色
	escapeinside=``, % 允许在代码块中使用 LaTeX 命令
	xleftmargin=2em, xrightmargin=2em, aboveskip=1em, % 设置代码块的边距
	framexleftmargin=2em, % 阴影框左边距
	showstringspaces=false
} 
\begin{document}
\maketitle
\newpage
\renewcommand{\contentsname}{目录} 
\tableofcontents
\newpage 
\section{信息的表示}
\subsection{整数的表示}
\subsubsection{整型的数据类型}
下面两个表格给出了C语言的各种整型在32位和64位两种程序下的大小以及范围。
\begin{table}[h!]
	\begin{center}
		\begin{tabular}{c|c|c|c}
			\textbf{类型名} & \textbf{占用字节数(Byte)} & \textbf{最小值} & \textbf{最大值} \\
			\hline
			char & 1 & -128 & 127\\
			\hline
			unsigned char & 1 & 0 & 255\\
			\hline
			short & 2 & -32768 & 32767\\
			\hline
			unsigned short & 2 & 0 & 65535\\
			\hline
			int & 4 & -2147483648 & 2147483647\\
			\hline
			unsigned int & 4 & 0 & 4294967295\\
			\hline
			long & 4 & -2147483648 & 2147483647\\
			\hline
			unsigned long & 4 & 0 & 4294967295\\
			\hline
			int32\_t & 4 & -2147483648 & 2147483647\\
			\hline
			uint32\_t & 4 & 0 & 4294967295\\
			\hline
			int64\_t & 8 & -9223372036854775808 & 9223372036854775807\\
			\hline
			uint64\_t & 8 & 0 & 18446744073709551615\\
			\hline
		\end{tabular}
	\caption{32位程序的C语言数据类型大小和范围}
	\end{center}
\end{table}

\begin{table}[h!]
	\begin{center}
		\begin{tabular}{c|c|c|c}
			\textbf{类型名} & \textbf{占用字节数(Byte)} & \textbf{最小值} & \textbf{最大值} \\
			\hline
			char & 1 & -128 & 127\\
			\hline
			unsigned char & 1 & 0 & 255\\
			\hline
			short & 2 & -32768 & 32767\\
			\hline
			unsigned short & 2 & 0 & 65535\\
			\hline
			int & 4 & -2147483648 & 2147483647\\
			\hline
			unsigned int & 4 & 0 & 4294967295\\
			\hline
			long & 8 & -9223372036854775808 & 9223372036854775807\\
			\hline
			unsigned long & 8 & 0 & 18446744073709551615\\
			\hline
			int32\_t & 4 & -2147483648 & 2147483647\\
			\hline
			uint32\_t & 4 & 0 & 4294967295\\
			\hline
			int64\_t & 8 & -9223372036854775808 & 9223372036854775807\\
			\hline
			uint64\_t & 8 & 0 & 18446744073709551615\\
			\hline
		\end{tabular}
	\caption{64位程序的C语言数据类型大小和范围}
	\end{center}
\end{table}
可以看出每种整型都分为有符号和无符号两种, 了解它们的编码方式就可以快速得到它们的范围.
\subsubsection{无符号数的编码}
假设一个$\omega$位的01向量是一个无符号整数的编码, 记$$\boldsymbol{x}=[x_{\omega-1},x_{\omega-2},\dots,x_0]$$
其中$x_i=0\text{ or }1,\quad i=0,1,\dots,\omega-1$. 我们用记号$\mathrm{B2U}_{\omega}(\boldsymbol{x})$表示$\boldsymbol{x}$所代表的无符号整数的值, 那么就有
$$\mathrm{B2U}_{\omega}(\boldsymbol{x})=\sum_{i=0}^{\omega-1}x_i2^i$$
在数学上很容易证明无符号整数与编码按这种规则的对应是双射(单射且满射). \par 
下面给出一些例子:
$$\begin{aligned}
&\mathrm{B2U}_{4}([0101])=0\cdot2^3+1\cdot2^2+0\cdot2^1+1\cdot2^0 = 5\\
&\mathrm{B2U}_{4}([1111])=1\cdot2^3+1\cdot2^2+1\cdot2^1+1\cdot2^0 = 15\\
&13=1\cdot2^3+1\cdot2^2+0\cdot2^1+1\cdot2^0 =\mathrm{B2U}_{4}([1101])
\end{aligned}$$
不难得出结论, $\omega$位编码表示的无符号整型的最小值是0, 对应编码为$\omega$个0,  最大值是$2^{\omega}-1$, 对应编码为$\omega$个1.
\subsubsection{有符号数的补码(two's-complement)编码}
假设一个$\omega$位的01向量是一个有符号整数的补码编码, 记$$\boldsymbol{x}=[x_{\omega-1},x_{\omega-2},\dots,x_0]$$
其中$x_i=0\text{ or }1,\quad i=0,1,\dots,\omega-1$. 我们用记号$\mathrm{B2T}_{\omega}(\boldsymbol{x})$表示$\boldsymbol{x}$在补码编码规则下所代表的有符号整数值, 那么就有
$$\mathrm{B2T}_{\omega}(\boldsymbol{x})=-x_{\omega-1}2^{\omega-1}+\sum_{i=0}^{\omega-2}x_i2^i$$
最高位$x_{\omega-1}$称为符号位, 权重为$-2^{\omega-1}$, 容易发现符号位为1时值为负数, 反之为正数,  并且补码表示有符号整数也是双射.\par 
下面给出一些例子:
$$\begin{aligned}
&\mathrm{B2T}_{4}([0101])=-0\cdot2^3+1\cdot2^2+0\cdot2^1+1\cdot2^0 = 5\\
&\mathrm{B2T}_{4}([1111])=-1\cdot2^3+1\cdot2^2+1\cdot2^1+1\cdot2^0 = -1\\
&-3=-1\cdot2^3+1\cdot2^2+0\cdot2^1+1\cdot2^0 =\mathrm{B2T}_{4}([1101])阿
\end{aligned}$$
补码编码的$\omega$位有符号数最大值为$2^{\omega-1}-1$, 对应的编码为$[0,1,\dots,1]$, 最小值为$-2^{\omega-1}$, 对应的编码为$[1,0,\dots,0]$. 我们也容易得到如下结论: 如果两个$\omega$位补码编码除了符号位都相同, 那么它们的大小相差$2^{\omega-1}$.\par 
下表给出一些同样的编码表示为无符号数和用补码规则表示有符号数的结果
\begin{table}[H]
	\begin{center}	
		\begin{tabular}{c|c|c|c}
			\textbf{hex} & \textbf{binary} & \textbf{unsigned int} &
			\textbf{signed int}\\
			\hline
			0xE & 1110 & 14 & -2\\
			\hline
			0x0 & 0000 & 0 & 0\\
			\hline
			0x5 & 0101 & 5 & 5\\
			\hline
			0x8 & 1000 & 8 & -8\\
			\hline
			0xD & 1101 & 13 & -3\\
			\hline
			0xF & 1111 & 15 & -1\\
			\hline
		\end{tabular}
	\end{center}
\end{table}
从表中可以注意到如下规律: 如果编码最高位是0, 那么无符号和有符号表示的整数值相同; 如果编码最高位是1, 那么它表示的无符号数与其补码规则表示的有符号数的和是$2^{\omega}$, 这个规律在数学上的证明也是显然的.\par 
最后, 我们可以证明, 如果两个使用补码表示符号数的等长编码的每一位都不同, 那么它们表示的符号数的和为-1, 这也就是说对于一个符号数x, 将它的二进制补码表示按位取反, 即可得到-x-1的补码表示. 
\subsubsection{反码与原码}
\textbf{反码}(Ones' Complement)的规则为$$\mathrm{B2O}_{\omega}(\boldsymbol{x})=-x_{\omega-1}(2^{\omega-1}-1)+\sum_{i=0}^{\omega-2}x_i2^i$$
\par
\textbf{原码}(Sign-Magnitude)的规则为$$\mathrm{B2S}_{\omega}(\boldsymbol{x})=(-1)^{x_{\omega-1}}\cdot(\sum_{i=0}^{\omega-2}x_i2^i)$$
\par 
这两种有符号整数的编码方式已经很少采用了, 其中一个原因是非双射性, 对于数字0, 除了使用$[0,\dots,0]$表示外, 反码还可以使用$[1,\dots,1]$表示, 原码还可以使用$[1,0,\dots,0]$表示. 现代机器几乎全部使用补码, 所以之后的符号数表示我们都默认是使用补码的表示.
\subsubsection{无符号和有符号的类型转换}
C语言中的强制类型转换本质上就是对于同一个编码的不同解读, 比如将32位有符号整数-1强制转换为无符号整数
\lstset{language=c++}
\begin{lstlisting}
int a = -1;
unsigned int b = a;
\end{lstlisting}
输出变量b, 结果为4294967295. 因为-1的32位补码表示为0xFFFFFFFF, 将其解析为无符号数便得到了32位的最大无符号数.\par 
反过来, 定义一个无符号数, 转换为有符号数
\lstset{language=c++}
\begin{lstlisting}
unsigned int a = 4294967294u;
int b = a;
\end{lstlisting}
a的编码为0xFFFFFFFE, 解析为有符号数变为-2.\par 
我们可以得到下列结论:
\begin{itemize}
	\item $\omega$位的有符号数x, 强制转换为$\omega$位的无符号数y, 则有
	$$y=\left\{
	\begin{aligned}
	&x, &x\geq0,\\
	&x+2^{\omega}, &x<0
	\end{aligned}
	\right.$$
	\item $\omega$位的无符号数x, 强制转换为$\omega$位的有符号数y, 则有
	$$y=\left\{
	\begin{aligned}
	&x, &x\leq2^{\omega-1}-1,\\
	&x-2^{\omega}, &x\geq2^{\omega-1}
	\end{aligned}
	\right.$$
\end{itemize}
C语言中对含有无符号的表达式运算有一些奇特的处理, 例如对于有符号和无符号的数字比较大小, C语言会默认将有符号数强制类型转换为无符号数, 考虑以下代码:
\lstset{language=c++}
\begin{lstlisting}
int max_int = 0x7FFFFFFF; // 2147483647
int min_int = -max_int - 1;
cout << "-1 < 0u ?" << ((-1 < 0u) ? "True" : "False") << endl;
cout << max_int << "u > " << min_int << " ? "
<< ((2147483647u > min_int) ? "True" : "False") << endl;
cout << max_int << " > int(2147483648u) ? "
<< (max_int > int(2147483648u) ? "True" : "False") << endl;
\end{lstlisting}
其输出为
\lstset{language=c++}
\begin{lstlisting}
-1 < 0u ? False
2147483647u > -2147483648 ? False
2147483647 > int(2147483648u) ? True
\end{lstlisting}
出现这样的结果就是C语言的强制类型转换导致的, 在表达式$-1 < 0u$中, 编译器会将-1强行转换为无符号数再比较, -1的补码为0xFFFFFFFF, 解析为无符号数会变为最大的无符号数. 表达式$2147483647u > -2147483648$中, 右侧数字是最小的32位符号数, 补码为0x80000000, 解析为无符号数变为$2^{31}$, 大于左侧的$2^{31}-1$. 最后一个表达式$2147483647 > int(2147483648u)$, 右侧数字由无符号转为有符号, 编码是0x80000000, 转为符号数变为最小32位符号数. 
\subsubsection{整型编码的扩展}
扩展指将低位数补全位数变为高位数, 这样的操作显然是安全可控的. 对于无符号数, 扩展只需要高位补0即可, 例如四位无符号数0x8, 扩展到八位, 十六位, 三十二位分别为0x08, 0x0008,0x00000008. 
\subsubsection{整型编码的截断}
\subsection{整数运算}
\subsubsection{无符号加法}

\end{document}
